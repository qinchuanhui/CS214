\documentclass[12pt,a4paper]{article}
\usepackage{ctex}
\usepackage{amsmath,amscd,amsbsy,amssymb,latexsym,url,bm,amsthm}
\usepackage{epsfig,graphicx,subfigure}
\usepackage{enumitem,balance}
\usepackage{wrapfig}
\usepackage{mathrsfs,euscript}
\usepackage[usenames]{xcolor}
\usepackage{hyperref}
\usepackage[vlined,ruled,linesnumbered]{algorithm2e}
\usepackage{array}
\hypersetup{colorlinks=true,linkcolor=black}

\newtheorem{theorem}{Theorem}
\newtheorem{lemma}[theorem]{Lemma}
\newtheorem{proposition}[theorem]{Proposition}
\newtheorem{corollary}[theorem]{Corollary}
\newtheorem{exercise}{Exercise}
\newtheorem*{solution}{Solution}
\newtheorem{definition}{Definition}
\theoremstyle{definition}

\renewcommand{\thefootnote}{\fnsymbol{footnote}}

\newcommand{\postscript}[2]
 {\setlength{\epsfxsize}{#2\hsize}
  \centerline{\epsfbox{#1}}}

\renewcommand{\baselinestretch}{1.0}

\setlength{\oddsidemargin}{-0.365in}
\setlength{\evensidemargin}{-0.365in}
\setlength{\topmargin}{-0.3in}
\setlength{\headheight}{0in}
\setlength{\headsep}{0in}
\setlength{\textheight}{10.1in}
\setlength{\textwidth}{7in}
\makeatletter \renewenvironment{proof}[1][Proof] {\par\pushQED{\qed}\normalfont\topsep6\p@\@plus6\p@\relax\trivlist\item[\hskip\labelsep\bfseries#1\@addpunct{.}]\ignorespaces}{\popQED\endtrivlist\@endpefalse} \makeatother
\makeatletter
\renewenvironment{solution}[1][Solution] {\par\pushQED{\qed}\normalfont\topsep6\p@\@plus6\p@\relax\trivlist\item[\hskip\labelsep\bfseries#1\@addpunct{.}]\ignorespaces}{\popQED\endtrivlist\@endpefalse} \makeatother

\begin{document}
\noindent

%========================================================================
\noindent\framebox[\linewidth]{\shortstack[c]{
\Large{\textbf{Lab11-NP Reduction}}\vspace{1mm}\\
CS214-Algorithm and Complexity, Xiaofeng Gao, Spring 2020.}}
\begin{center}
\footnotesize{\color{red}$*$ If there is any problem, please contact TA Shuodian Yu. }

\footnotesize{\color{blue}$*$ Name:Yulong Hui  \quad Student ID:518030910059 \quad Email: qinchuanhuiyulong@sjtu.edu.cn}
\end{center}
\begin{enumerate}
	\item What is the ``certificate'' and ``certifier'' for the following problems?
	\begin{enumerate}
		\item \emph{ZERO-ONE INTEGER PROGRAMMING}: Given an integer $m \times n$ matrix $A$ and an integer $m$-vector $b$, is there an integer $n$-vector $x$ with elements in the set $\{0, 1\}$ such that $Ax \leq b$.
		\item \emph{SET PACKING}: Given a finite set $U$, a positive integer $k$ and several subsets $U_1, U_2, \ldots, U_m$ of $U$, is there $k$ or more subsets which are disjoint with each other?
		\item \emph{STEINER TREE IN GRAPHS}: Given a graph $G=(V,E)$, a weight $w(e)\in Z_0^{+}$ for each $e\in E$, a subset $R \subset V$, and a positive integer bound $B$, is there a subtree of $G$ that includes all the vertices of $R$ and such that the sum of the weights of the edges in the subtree is no more than $B$.
	\end{enumerate}
	 
	 ~\\
	 \textbf{Solution.}
	 \begin{enumerate}
	 	\item 
	 	\textbf{Certificate.} An integer $n$-vector $x$ with elements in the set $\{0,1\}$.
	 	
	 	\textbf{Certifier.} Check that $n$-vector $x$ satisfies: $Ax \leq b$.

	 
		
		\item
		\textbf{Certificate.} A collection  set $C$ of subsets $U_1,U_2,...,U_m$, whose cardinality is not less than $k$.
		
		\textbf{Certifier.} Check that the sets in  $C$ are disjoint with each other.
		
		\item
		\textbf{Certificate.} A subtree of $G$ that includes all the vertices of R.
		
		\textbf{Certifier.} Check that the sum of the weights of the edges in the subtree is no more than $B$.
	 \end{enumerate}
	 
	 
	\item Algorithm class is a democratic class. Denote class as a finite set $S$ containing every students. Now students decided to raise a student union $S' \subseteq S$ with $|S'|\leq K$ .
	
	As for the members of the union, there are many different opinions. An opinion is a set $S_o\subseteq S$. Note that number of opinions has nothing to do with number of students.
	
	The question is whether there exists such student union $S' \subseteq S$ with $|S'|\leq K$, that $S'$ contains at least one element from each opinion. We call this problem \emph{ELECTION} problem, prove that it is NP-complete.

	\begin{proof}
		It is known to us that VERTEX-COVER problem is NP-complete, so we can just finish this proof by proving: VERTEX-COVER$\; \leq_P\;$ELECTION. 
		
		Assume any graph $G=(V,E)$, and let $S$ have the same meaning as $V$. Then students in $S$ equal to the vertices in $V$. Let every edge $e_i$ (contains 2 vertices) in $V$ equals to an opinion $S_i$, which means there are always 2 students in every opinion. Therefore, if we need to solve VERTEX-COVER problem and find a subset of $V$ which have at least one element in every edge $e_i$, we can transform it to a special ELECTION problem and find a student union $S'\subseteq S$ which contains at least one element in each opinion $S_i$.
		
		Then we has proved:  VERTEX-COVER$\; \leq_P\;$ELECTION, so ELECTION problem is NP-complete.
		  
	
	\end{proof}

	\item Not-All-Equal Satisfiability (NAE-SAT) is an extension of SAT where every clause has at least one true literal and at least one false one. NAE-$3$-SAT is the special case where each clause has exactly $3$ literals. Prove that NAE-$3$-SAT is NP-complete. (Hint : reduce $3$-SAT to NAE-$k$-SAT for some $k > 3$ at first)
	    \begin{proof}
       First, we can reduce the 3-SAT problem to a NAE-4-SAT problem:
       
       Assume one clause as $(x_1\bigvee x_2 \bigvee x_3) $, then it equals to a NAE-4 clause $N_4(x_1,x_2,x_3,F)$. In the NAE-4 clause, the result will be true iff it has at least one true value among $x_1,x_2,x_3$, which can solve the 3-SAT prblem. Therefore, we can get: 3-SAT  $\leq_P$ NAE-4-SAT.
       
       Then we can reduce the NAE-4-SAT problem to a NAE-3-SAT problem:
       
       Assume one clause as $N_4(x_1,x_2,x_3,x_4)$, and construct a NAE-3 format as follows: ($s$ is a binary variable) $$N=N_3(x_1,x_2,s)\bigwedge N_3(x_3,x_4,
       \overline s)$$  It's easy to see that if $x_1,x_2,x_3,x_4$ are  all equal, no matter what value the $s$ is, $N$ is always false. Then if $x_1,x_2,x_3,x_4$ are not all equal, we can always find a value for $s$ and make $N$ true. Therefore, we can sovle the NAE-4-SAT problem by constructing a such NAE-3 format formula. Then, we can get: NAE-4-SAT $\leq_P$ NAE-3-SAT.
       
       So, 3-SAT  $\leq_P$ NAE-4-SAT $\leq_P$ NAE-3-SAT, and NAE-3-SAT is NP-complete.
       
    \end{proof}

	\item In the Lab10, we have introduced Minimum Constraint Data Retrieval Problem (MCDR). Prove that MCDR (Version $1$ or $2$) is NP-complete. (Hint : reduce from VERTEX-COVER or $3$-SAT)
	   \begin{proof}
 		We should make some definistions clear:
 		
 		MCDR-V1 means: Does there exist a scenario whose switch-number $\leq k$,  while access-latency is no more than $t$.
 		
 		MCDR-V2 means: Does there exist a scenario whose access-latency $\leq k$,  while switch-number is no more than $h$.
 		
 		At first, we will prove MCDR-V1 by reducing from VERTEX-COVER:
 		
 		Assume a graph $G=(V,E)$, if there is a vertex $v_i$ corresponding to several edges $e_1,e_2,..e_j$, then we can construct a channel
 		$c_i$ with continuous data-items $d_1,d_2,...d_j$ in it. That's to say a channel equals to a vertex, and a data-item  equals to a edge.
 		
 		Then we can transform the VERTEX-COVER problem (is there a vertex set of size $\leq k$ which can cover all the edges ) to a MCDR-V1 problem (is there a scenario of switch-time $\leq (k-1)$ which can cover all the data-items). Specially, in this case the constraint $t$ of access-time is infinite, and we just need to find the smallest switch-time which equals to "the size of vertex-set - 1".
 		
 		Then, we will prove MCDR-V2 by reducing from VERTEX-COVER:
 		
 			Assume a graph $G=(V,E)$, and calculate the value of $|E|$ in poly-time. For vertex $v_i$ corresponding to several edges $e_1,e_2,..e_j$, we construct $a_i$ channels with these $d_1,d_2,..d_j$. These channels should make sure that every $d_k$ can be at the $1st,2nd,...jth$ places,
 			so $a_i\leq j^2$, which follows poly-time.
 			
 			Then, if we want to solve the VERTEX-COVER problem (is there a vertex set of size $\leq k$ which can cover all the edges ). We can start at one channel, after download all the items in this channel, we can transform to another one. Because $d_k$ can be at the $1st,2nd,...jth$ places, we will be able to find a proper channel, in which the item $d_k$ is just after a switch and we don't need to wait other items before getting $d_i$. 
 			
 			 Then, if we find a scenario of access-time $\leq (|E|+k-1)$, it means we find a vertex cover set of size $\leq k$. Specially, in this case  the constraint on switch-time $h$ is infinite.
 			 
 			 Therefore we have proved: VERTEX-COVER $\leq_P$ MCDR-V1, MCDR-V2. Then MCDR-V1 and MCDR-V2 are NP-complete. 
 		
 		
         \end{proof}
\end{enumerate}

\textbf{Remark:} Please include your .pdf, .tex files for uploading with standard file names.




%========================================================================
\end{document}
